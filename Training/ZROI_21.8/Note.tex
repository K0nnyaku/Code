\documentclass[UTF8]{ctexart}
\author{我}
\title{测试}
\begin{document}
## 集合幂级数

#### 定义

集合幂级数即为形如 $\sum\limits_{i=0}^{2^n-1}a_ix^i$​ ,其中二进制数 $i$​ 表示一个 $\{1,2,\ldots,n\}$​ 的一个子集。

#### 一些基本操作

很多题目要做的就是以下这几种操作:

1. 高维前缀和: $c_i=\sum\limits_j [j\or i =i] a_j$ 。

2. 高维后缀和: $c_i=\sum\limits_j[j\and i=i] a_j$ 。

3. 或卷积: $c_i=\sum\limits_j\sum\limits_k [j \or k=i] a_jb_k$​ 。

4. 与卷积: $c_i=\sum\limits_j\sum\limits_k[j\and k=i]a_jb_k$ 。

5. 异或卷积: $c_i=\sum\limits_j\sum\limits_k [j\oplus k=i] a_jb_k$​ 。

6. 子集卷积: $c_i=\sum\limits_j\sum\limits_k[j\and k=\emptyset,j\or k=i] a_jb_k$​ 。
7. 子集卷积 exp : $c_i=\sum\limits_{i_1,i_2,\ldots,i_k} [|i_1|+|i_2|+\cdots+|i_k|=|i|,i_1\or i_2\or \cdots \or i_k=i]a_{i_1}a_{i_2}\cdots a_{i_k}$​ 。​
8. 多项式复合集合幂级数: $c=\sum\limits_{i=0}^n f_ia^i$ ,其中 $a^i$ 为子集卷积。

如果上述操作我们暴力求解,复杂度均为 $O(2^{2n})$ 。我们依次考虑这些问题如何在比较好的时间复杂度内解决。

1. 高维前缀和

   我们考虑怎么做一维前缀和: 

   ```
   for(int i=1;i<=n;i++) a[i]+=a[i-1];
   ```

   二维前缀和呢:

   ```
   for(int i=1;i<=n;i++) for(int j=1;j<=m;j++) a[i][j]+=a[i][j-1];
   for(int i=1;i<=n;i++) for(int j=1;j<=m;j++) a[i][j]+=a[i-1][j];
   ```

   我们以此类推, $n$​ 维前缀和即为每次枚举一维,将这一维上做一次前缀和:

   ```
   for(int i=0;i<n;i++)
   	for(int j=0;j<(1<<n);j++)
   		if(j&(1<<i)) a[j]+=a[j^(1<<i)];
   ```

   这样我们就以 $O(2^nn)$ 的时间复杂度解决了高维前缀和问题。

   我们称这个高维前缀和过程为**快速莫比乌斯变换**,即**FMT**。

2. 高维后缀和

   我们效仿高维前缀和,每次枚举一维做后缀和即可。

   ```
   for(int i=0;i<n;i++)
   	for(int j=0;j<(1<<n);j++)
   		if(j&(1<<i)) a[j^(1<<i)]+=a[j];
   ```

3. 或卷积

   我们定义 $a$ 经过 FMT 后得到的集合幂级数为 $A$ , $b$ 经过 FMT 后得到的集合幂级数为 $B$ , $c$ 经过 FMT 后得到的集合幂级数为 $C$ ,我们容易发现 $C_i=A_iB_i$ 。

   所以我们只需要对 $a,b$ 分别做一次 FMT ,并将对应位相乘后做一次 FMT 的逆变换即可。

   而 FMT 的逆变换显然就是把刚才的过程反过来,即为:

   ```
   for(int i=0;i<n;i++)
   	for(int j=0;j<(1<<n);j++)
   		if(j&(1<<i)) a[j]-=a[j^(1<<i)];
   ```

   这样就求出了 $c$ 的所有系数,在 $O(2^nn)$ 的时间复杂度解决了或卷积。

4. 与或卷积类似,我们对 $a,b$​ 分别做一次高维后缀和,并将对应位相乘后做一次高维后缀和的逆运算即可。

   时间复杂度 $O(2^nn)$ 。

5. 异或卷积

   我们定义一个算子 $FWT(a)$ ,其中 $a$ 和 $FWT(a)$​ 都是一个集合幂级数。 $FWT(a)_i=\sum\limits_{j=0}^{2^n-1}(-1)^{|i\and j|} a_j$ 。

   我们想要说明如果 $a$ 和 $b$ 异或卷积后的结果为 $c$ ,那么有 $FWT(c)_i=FWT(a)_i\cdot FWT(b)_i$ 。

   证明:
   $$
   FWT(c)_i=\sum\limits_{j=0}^{2^n-1}(-1)^{|i\and j|}c_j\\
   =\sum\limits_{j=0}^{2^n-1}(-1)^{|i\and j|}\sum\limits_{k=0}^{2^n-1}\sum\limits_{l=0}^{2^n-1}[k\oplus l=j]a_kb_l\\
   =\sum\limits_{k=0}^{2^n-1}\sum\limits_{l=0}^{2^n-1}(-1)^{|(k\oplus l)\and i|}a_kb_l\\
   =\sum\limits_{k=0}^{2^n-1}\sum\limits_{l=0}^{2^n-1}(-1)^{|k\and i|}a_k\cdot (-1)^{|l\and i|} b_l\\
   =(\sum\limits_{k=0}^{2^n-1}(-1)^{|k\and i|}a_k)(\sum\limits_{l=0}^{2^n-1}(-1)^{|l\and i|}b_l)\\
   =FWT(a)_i\cdot FWT(b)_i
   $$
   所以我们要做的事情就是分别对 $a,b$ 求出 $FWT$ 后对应相乘再做 $FWT$ 的逆变换即可。

   和高维前缀和相似,我们对每一位依次考虑。对于第 $i$​ 位和一个不包含 $i$​ 的集合 $S$​​ ,设 $x=a_S,y=a_{S+2^i}$​ ,则有新的 $a_S=x+y,a_{S+2^i}=x-y$​ 。这样我们就以 $O(2^nn)$ 的时间复杂度求出了 $FWT$​ 。

   ```
   for(int i=1;i<(1<<n);i<<=1){
   	for(int j=0;j<(1<<n);j+=(i<<1)){
   		for(int k=j;k<j+i;k++){
   			int x=a[k],y=a[k+i];
   			a[k]=x+y; a[k+i]=x-y;
   		}
   	}
   }
   ```

   $FWT$​ 的逆运算直接对每一位做逆操作即可。  

   时间复杂度 $O(2^nn)$ 。

6. 子集卷积

   考虑如果我们做普通的或卷积,那么会有一些 $[j\and k\neq \emptyset,j\or k=i]$ 的 $(j,k)$ 对贡献到 $i$ 上,所以我们不能直接做或卷积。

   但注意到 $[j\and k=\emptyset,j\or k=i]$ 的条件等价于 $[|j|+|k|=i,j\or k=i]$ ,所以我们可以将所有集合按元素个数分组,将第 $x$ 组和第 $y$ 组或卷积得到的集合幂级数中元素个数恰为 $x+y$ 的部分贡献到最终答案中。

   如果我们对所有 $O(n^2)$​ 对 $(x,y)$​​ 暴力做或卷积,复杂度会是 $O(2^nn^3)$ 。但我们发现这样做对每一组操作了 $n$ 次 $FWT$ ,这是多余的操作。所以可以事先算出所有组的 $FWT$ ,然后对所有 $O(n^2)$ 对 $(x,y)$ 的 $FWT$ 数组直接对应位相乘加到答案,最后再把答案的每个组用 $FWT$ 的逆运算操作回去即可。

   时间复杂度 $O(2^nn^2)$​ 。

7. 子集卷积 exp

   考虑按最高位分组,每一组中最多选择一个,所以答案即为所有组子集卷积后的答案。考虑从低位到高位依次合并,合并第 $i$​ 位时的复杂度为 $O(2^ii^2)$ ,总复杂度为 $\sum\limits_{i=1}^n O(2^ii^2)=O(2^nn^2)$​ 。

8. 仍然按最高位分组,且每一组中最多选择一个,从低位到高位合并,但我们要记录当前还有几个要选。记 $G_{i,j}$ 表示合并完前 $i$ 组后还需要再添加 $j$ 个的方案数。初始即为 $G_{0,i}=f_i$ ,每次添加一组即为 $G_{i,j}=G_{i-1,j}+G_{i-1,j-1}F_i$ ,其中 $F_i$ 表示第 $i$ 组形成的集合。总复杂度为 $\sum\limits_{i=1}^nO(2^ii^2(n-i))=O(2^nn^2)$ 。

   时间复杂度 $O(2^nn^2)$ 。

#### 一些例题

gym103202M

gym103109K

CF662C

AGC43C

WC2018 州区划分

P5933 带边权连通图个数

loj154

\end{document}
